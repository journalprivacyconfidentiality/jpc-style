\documentclass{jpc} %%% last changed 2014-08-20

% JPC Layouting Macros
% THESE ARE ADDED BY THE EDITORIAL TEAM - NO NEED TO SET HERE
%\newcommand{\doisuffix}{v0.i0.999}
% \jpcheading{vol}{issue}{year}{notused}{subm}{publ}{rev}{spec_iss}{title}
%\jpcheading{0}{0}{2000}{}{Mar.~20, 2017}{Jun.~22, 2018}{}{Special issue}
%%% last changed 2014-08-20

%% mandatory lists of keywords 
\keywords{MANDATORY list of keywords}

%% read in additional TeX-packages or personal macros here:
%% e.g. \usepackage{tikz}
%\usepackage{hyperref}
\usepackage{natbib} 
\usepackage[ruled]{algorithm2e}
%%\input{myMacros.tex}
%% define non-standard environments BEYOND the ones already supplied 
%% here, for example
\theoremstyle{plain}\newtheorem{satz}[thm]{Satz} %\crefname{satz}{Satz}{S\"atze}
%% Do NOT replace the proclamation environments lready provided by
%% your own.

\def\eg{{\em e.g.}}
\def\cf{{\em cf.}}

%% due to the dependence on amsart.cls, \begin{document} has to occur
%% BEFORE the title and author information:

\begin{document}

\title[Instructions]{Instructions for Authors:\\How to prepare papers
  for JPC using \MakeLowercase{\texttt{jpc.cls}}\rsuper*\\ 
  2018-06-20}
\titlecomment{{\lsuper*}OPTIONAL comment concerning the title, \eg, 
  if a variant or an extended abstract of the paper has appeared elsewhere.}

\author[A.~Name1]{Alice Name1}	%required
\address{address 1}	%required
\email{name1@email1}  %optional
%\thanks{thanks 1, optional.}	%optional

\author[B.~Name2]{Bob Name2}	%optional
\address{address2; addresses should initially be duplicated, even if
  authors share an affiliation}	%optional
\email{name2@email2; ditto for email addresses}  %optional
\thanks{thanks 2, optional.}	%optional

\author[C.~Name3]{Carla Name3}	%optional
\address{address 3}	%optional
\urladdr{name3@url3\quad\rm{(optionally, a web-page can be specified)}}  %optional
\thanks{thanks 3, optional.}	%optional

%% etc.

%% required for running head on odd and even pages, use suitable
%% abbreviations in case of long titles and many authors:

%%%%%%%%%%%%%%%%%%%%%%%%%%%%%%%%%%%%%%%%%%%%%%%%%%%%%%%%%%%%%%%%%%%%%%%%%%%

%% the abstract has to PRECEDE the command \maketitle:
%% be sure not to issue the \maketitle command twice!

\begin{abstract}
  \noindent The abstract has to precede the maketitle command.  Be
  sure not to issue the maketitle command twice!  In the abstract,
  mathematical expressions must be kept to the absolute minimum.
  Otherwise it should consist of plain ASCII text, without
  \TeX-commands, including explicit references using the
  \texttt{\textbackslash cite} command.  Presently we are not able to
  automatically extract an abstract containing such data and reliably
  turn it into html code.  If you cannot meet these criteria, it is
  your responsibility to provide us with an html-version of your
  abstract.  Please keep the abstract fairy short to prevent it from
  spilling onto the second page!
\end{abstract}

\maketitle

%% start the paper here:
\section*{Introduction}\label{S:one}

  The Journal of Privacy and Confidentiality is a small journal, run
  by scientists like you who devote their time and effort to make this
  a world-class open access journal that is free of cost for readers
  as well as authors.  To minimize the extra work for the layout
  editor and to ensure smooth and fast publication of accepted
  articles, authors are asked to strictly adhere to the instructions
  for preparing their final version given in this document, which
  takes the form of a sample paper.

  These instructions rely on the experience of multiple editors running this
  Journal and others for several years, in particular the efforts of the editors of Logical Methods in Computer Science (LMCS, \url{https://lmcs.episciences.org/}), whose \LaTeX\  class (and this "instructions.tex") we used as a basis for ours.  The instructions here address the most time-consuming
  aspects of getting articles into publishable shape. 
  
\subsection*{A note about Word documents}
For those authors who are unfamiliar with \LaTeX, we will provide some limited assistance. However, we strongly encourage you to try out modern \LaTeX\  tools, such as Overleaf (\url{https://www.overleaf.com}), which make this quite easy. 


\subsection*{A note about Wordperfect documents}
Really?

\subsection*{\TeX-nical matters}

  Please be aware that the class-file {\bf jpc.cls} supplied to 
  authors will be replaced by the Journal's master class-file before
  publication of your article.  Hence it is not necessary, and is in
  fact counterproductive, to emulate the appearance of published
  articles by means of your personal macros.  Submissions not using
  {\bf jpc.cls} will be returned to the authors, as the reformatting
  that usually results from changing the class-file is usually too
  extensive and requires the original authors' intervention.

  What authors \emph{can} do to help the layout editor is to make
  their \TeX- source compatible with the {\bf hyperref}-package, which
  is included by the master class-file.  In particular, care should be
  taken to use the \texttt{\textbackslash texorpdfstring} macro for
  mathematical expressions in section or subsection headings (see, for
  instance, \href{http://www.fauskes.net/nb/latextips/#hyperref}{this
    explanation}).  

  Authors \textbf{must not} (1) use unsupported fonts (like the
  \texttt{times}-package or the \texttt{txfonts}-package), (2) change
  the numbering style for theorems and definitions and the like, \eg,
  by redefining the already provided proclamation environments for
  Theorems, Propositions, Lemmata, Corollaries etc.\ (you can add
  further environments, but those should comply with the default
  numbering style), and (3) use the \texttt{\textbackslash sloppy}
  option globally.  If it is impossible to achieve good line breaks by
  other means once the article is finished (reformulating a sentence,
  changing the word order, etc.), one can use \texttt{\textbackslash
    sloppy} as a last resort \emph{locally} in a paragraph.

  Using lengthy mathematical expressions inside running text can lead
  to ugly breaks within formulae, even without producing overfull
  hboxes.  If this is a persistent problem in your paper, please
  consider using more displayed formulae, or changing your notation.

  The use of different macro-packages for the purpose of creating
  diagrams or other graphical displays is strongly discouraged.  In
  the past that has led to papers that required different ways of
  processing to display the graphics of one type or the other, but
  could not easily be made to correctly display both types of graphics
  simultaneously.  As a rule of thumb, as long as {\bf pdflatex}
  correctly processes your paper, you should be in good shape. (Users
  of the {\bf pstricks}-package and those used to including external
  eps-files should transform the resulting PostScript files to pdf.)

  Please be aware that the proofs may display different vertical
  spacing in general, in particular different page breaks than the
  version originally submitted.  If adjustments are deemed to be
  necessary by the authors, they can be implemented on the basis of
  the proofs, in collaboration with the layout editor, as a last step
  before final publication.
 
\subsection*{Matters of convenience}

  Please submit {\bf only one file} containing the
  TeX-source of your paper!   Of course,
  we understand that separating a TeX source into several files has
  advantages during the creation of a paper, but please combine all
  parts into a single file for your submission. 
  
  The following are exceptions to the above rule:
  \begin{itemize}
  	\item Your personal macros can  be contained in a separate file.
  	\item You can use a single \texttt{.bib} file for bibliographic references. It should only include references actually used.
  	\item External
  graphics should of course be separately provided, with highest possible resolution. All graphics should be in a separate dedicated subdirectory (\eg ``graphics'' or ``figures'').
\end{itemize}

\subsection*{Matters of style}

  See the \href{https://journalprivacyconfidentiality.org/index.php/jpc/about/submissions}{website} for more extensive writing style suggestions.

\section{Multiple authors}

  In papers with multiple authors several points need to be mentioned.
  Do not worry about footnote signs that will link author $n$ to
  address $n$ and the optional thanks $n$.  This will be taken care of
  by the layout editor.  Even if authors share an affiliation and part
  of an email address, they should follow the strict scheme outlined
  above and list their data individually.  The layout editor will
  notice duplication of data and can then arrange for more
  space-efficient formatting.  Alternatively, Authors can write ``same
  data as Author n'' into some field to alert the layout editor.
  Unfortunately, so far we have not been able to devise a system that
  automatically takes care of these issues.  But once the layout
  editor is made aware of some duplication, he can take some fairly
  simple measures to adjust the format accordingly.  Placing the
  responsibility on the layout editor insures that these formatting
  issues are handled uniformly in different papers and that the
  authors do not have to second-guess the Journal's policy.

\section{Use of  Definitions and Theorems etc.}

  The numbering scheme for proclamations (Theorems, Definitions, etc.)
  uses the section number followed by the number of the current
  proclamation.  There are no different ``numbering threads'' for the
  various types of proclamations, as then the relative position of,
  \eg, Theorem 2.7 relative to, \eg, Definition 2.9 would not be clear.

\begin{defi}\label{D:first}
  This is a definition.
\end{defi}

  Please use the supplied proclamation environments (as well as
  LaTeX's cross-referencing facilities), or extend them in the spirit
  of the given ones, if necessary.
  Refrain from replacing the Journal's proclamation macros by your own
  constructs, especially do not change the numbering scheme: all
  proclamations are to be numbered consecutively!

\subsection{First Subsection}

  This is a test of subsectioning.  It works like numbering of
  paragraphs but is not linked with the numbering of theorems.


\begin{proof} You can use the familiar \texttt{\textbackslash
    begin\{proof\}}\dots\texttt{\textbackslash end\{proof\}}
  construction.  Please do not insert a blank line before
  \texttt{\textbackslash end\{proof\}}, as this moves the box to a new
  line.  

  In case a proof ends with an itemization,
  please issue the command \texttt{\textbackslash qedhere} at the end
  of the final item, \emph{before} calling \texttt{\textbackslash
    end\{enumerate\}} (or similar) and \texttt{\textbackslash end\{proof\}}.
  Otherwise the end-of-proof box is put on a separate line
  following the last item, which looks awkward, unless the last line
  is too full to accommodate the box.

  For options how to handle proofs ending in a displayed multi-line
  equation or formula, see
  \href{http://tex.stackexchange.com/questions/101929/qed-or-qedhere-at-the-end-of-split-environment}{this
    discussion}.
\end{proof}

\begin{cor}\label{C:big}
  If no proof is given, \texttt{jpc.cls} provides Paul Taylor's
  end-of-proof box \emph{\texttt{\textbackslash qed}} to conclude a
  proclamation (Theorem, Proposition, Lemma, Corollary).  Please do
  not redefine \emph{\texttt{\textbackslash qed}}!\qed
\end{cor}

\section{Algorithms}\label{S:algo}
Algorithms and pseudo-code should be straightforward using the \texttt{algorithm2e} package:


\begin{algorithm}[H]

 \KwData{The original data set $X=(x_1,...,x_p)$}
 \KwResult{A result data set $Z=(z_1,...,z_p)$}
 \For{each $i \in 2,..,p$}{
  1a. Model $x_i|x_1,...,x_{i-1}$ using some complex method\;
  \For{each $j \in 1,...,n$}{
   2a. Randomly select a quantile and choose the model corresponding to the quantile\;
   2b. From this model, generate $z_{j,i}$ for each observation from $z_{j,1},...,z_{j,i-1}$ and return as result data value for $x_{j,i}$\;
   } 
 }
 \caption{An iterative procedure to generate data}
 \label{alg:data}
\end{algorithm}

\noindent For instance, Alg.~\ref{alg:data} describes an algorithm. We suggest using the \texttt{ruled} option. 

\section{Itemization}\label{S:item}
  \texttt{JPC.cls} provides the familiar environments 
\begin{enumerate}
\item\texttt{\textbackslash
  begin\{itemize\}}\dots\texttt{\textbackslash end\{itemize\}} 
\item\texttt{\textbackslash
  begin\{enumerate\}}\dots\texttt{\textbackslash end\{enumerate\}}
  (see this listing)
\item\texttt{\textbackslash
    begin\{description\}}\dots\texttt{\textbackslash end\{description\}}
\end{enumerate}
  in a form based on the \texttt{enumitem}-package, version 3.5.2
  (please update, if you have an earlier version).  This offers
  considerable simplifications, both for authors and the Layout
  Editor.  Modifying the spacing of these environments is strongly
  discouraged.  If you wish to change the labels, please consult the
  documentation of the \texttt{enumitem}-package.  A simple example is
  found at the end of this document.

  When proclamations or proofs start with an itemization without
  preceding text, two possibilities exist:

\begin{thm}\label{T:m}\hfill  %% \hfill pushes the first item to a new line
\begin{enumerate}
\item
  Issuing an {\em\texttt{\textbackslash hfill}}-command before the
  beginning of the list environment will push the first item to a new
  line, like in this case.
\item
  This is the second item.
\end{enumerate}
\end{thm}

\proof\hfill  %% \hfill pushes the first item to a new line
\begin{enumerate}
\item
  The same behavior occurs in proofs; to start the first item on a
  new line an explicit \texttt{\textbackslash hfill}-command is necessary.
\item
  Citations can be formatted as you like, provided (i) the formatting is consistent and (ii) each citation begins with the last name of the first author, as in \cite{Abowd:Nissim:Skinner:2009}. Package \texttt{\textbackslash natbib} and a style such as \texttt{abbrvnat} work well. \qed
\end{enumerate}

  \noindent We strongly recommend using this variant since it produces
  rather orderly output.  The space-saving variant, in contrast, can
  look quite awful, \cf~Theorem \ref{T:en} below.  Please notice that
  this paragraph is not indented, since it is following a proclamation
  that ended with a list environment.  This can be achieved by
  starting the paragraph directly after the end of that environment,
  without inserting a blank line, or by explicit use of the
  noindent-command at the beginning of the paragraph.  The effect
  indentation may have after a list environment is demonstrated after
  the proof of Theorem \ref{T:en}. 
 
\begin{thm}\label{T:en} %% without \hfill the first item is indented

\begin{enumerate}%[\em(1)]
\item  
  Without the \emph{\texttt{\textbackslash hfill}}-command the first item
  starts in the same line as the title for the proclamation.
\item
  This may be useful when space needs to be conserved, but not in an
  electronic journal.
\end{enumerate}
\end{thm}

\proof %% without \hfill the first item is indented
\begin{enumerate}%[(1)]
\item
  As you can see, the second option produces a somewhat unpleasant effect.
\item
  Hence we would urge authors to use the first variant.  Perhaps a
  \TeX-guru can help us to make that the default, without the need for
  the \texttt{\textbackslash hfill}-command.\qed
\end{enumerate}

  Here we started a new paragraph without suppressing its
  indentation.  This adds to the rather disorienting appearance
  produced by not turning off the space-saving measures built into
  amsart.cls, on which this style is based.  Please do issue the
  \hbox{\textbackslash noindent} command in such situations, just as
  after the proof of Theorem \ref{T:m} above.

\section{Figures}
Figures should be used to help with exposition. Where possible, one or more figures should provide illustrations of main results. These figures may be split into multiple parts. The legend on these figures should have enough detail that someone can get an impression of what is being displayed just by looking at the figure.

\section*{Acknowledgment}
  \noindent The authors wish to acknowledge fruitful discussions with
  A and B.

%% in general the use of bibtex is encouraged

\bibliography{sample}
\bibliographystyle{abbrvnat}

\appendix
\section{}
  Here is a check-list to be completed before submitting the paper to
  JPC:
\begin{itemize}[label=$\triangleright$]
\item your submission uses the latest version of jpc.cls
\item the text of your submission is contained in a single file,
  except for macros and graphics
\item your graphics use only one format 
\item you have employed the Journal's original proclamation environments,
  or suitable extensions thereof 
\item you have loaded the hyperref package
\item you have \emph{not} loaded the times package
\item you have not routinely adjusted vertical spacing manually by issuing
  \texttt{\textbackslash vspace} or \texttt{\textbackslash vskip} commands
\item you have used the command \texttt{\textbackslash sloppy} only
  locally and in emergency cases
\item your displayed equations use the
  \texttt{\textbackslash[\dots\textbackslash]} construct
\item your abstract only contains as few math-expressions as possible and no
  references 
\end{itemize}

  This listing also shows how to override the default bullet $\bullet$
  of the \texttt{itemize}-envronment by a different symbol, in this
  case \texttt{\textbackslash triangleright}.
\end{document}
