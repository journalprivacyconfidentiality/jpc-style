\documentclass{jpcfinal} %%% last changed 2014-08-20

% JPC Layouting Macros =========================================
% THESE ARE ADDED BY THE EDITORIAL TEAM - NO NEED TO SET HERE
\newcommand{\doisuffix}{v0.i0.999}
% \jpcheading{vol}{issue}{year}{notused}{subm}{publ}{rev}{spec_iss}{title}
\jpcheading{0}{0}{2000}{}{Mar.~20, 2017}{Jun.~22, 2018}{}{Special issue}
%%% last changed 2018-06-29 =====================================

%% mandatory lists of keywords 
\keywords{MANDATORY list of keywords}

%% read in additional TeX-packages or personal macros here:
%% e.g. \usepackage{tikz}
%\usepackage{hyperref}
\usepackage{natbib} 
\usepackage{pdfpages}
%%\input{myMacros.tex}
%% define non-standard environments BEYOND the ones already supplied 
%% here, for example
\theoremstyle{plain}\newtheorem{satz}[thm]{Satz} %\crefname{satz}{Satz}{S\"atze}
%% Do NOT replace the proclamation environments lready provided by
%% your own.

\def\eg{{\em e.g.}}
\def\cf{{\em cf.}}

%% due to the dependence on amsart.cls, \begin{document} has to occur
%% BEFORE the title and author information:

\begin{document}

\title[Instructions]{Instructions for Authors:\\How to prepare papers
  for JPC using \MakeLowercase{\texttt{jpc.cls}}\rsuper*\\ 
  2018-06-20}
\titlecomment{{\lsuper*}OPTIONAL comment concerning the title, \eg, 
  if a variant or an extended abstract of the paper has appeared elsewhere.}

\author[A.~Name1]{Alice Name1}	%required
\address{address 1}	%required
\email{name1@email1}  %optional
%\thanks{thanks 1, optional.}	%optional

\author[B.~Name2]{Bob Name2}	%optional
\address{address2; addresses should initially be duplicated, even if
  authors share an affiliation}	%optional
\email{name2@email2; ditto for email addresses}  %optional
\thanks{thanks 2, optional.}	%optional

\author[C.~Name3]{Carla Name3}	%optional
\address{address 3}	%optional
\urladdr{name3@url3\quad\rm{(optionally, a web-page can be specified)}}  %optional
\thanks{thanks 3, optional.}	%optional

%% etc.

%% required for running head on odd and even pages, use suitable
%% abbreviations in case of long titles and many authors:

%%%%%%%%%%%%%%%%%%%%%%%%%%%%%%%%%%%%%%%%%%%%%%%%%%%%%%%%%%%%%%%%%%%%%%%%%%%

%% the abstract has to PRECEDE the command \maketitle:
%% be sure not to issue the \maketitle command twice!

\begin{abstract}
  \noindent The abstract has to precede the maketitle command.  Be
  sure not to issue the maketitle command twice!  Abstracts should include a brief summary of the main research findings. Avoid generic statements such as ``an example is presented''. Instead, minimal details about examples should be included, as in, ``we illustrate the benefit of our approach by analyzing data on employment from the UK Office of National Statistics''.  Avoid hyperlinks, explicit references using the
  \texttt{\textbackslash cite} command, and if possible, \LaTeX \ code.
  If you cannot meet these criteria, it is
  your responsibility to provide us with an HTML-version of your
  abstract.  Abstracts are limited to no more than 300 words.
\end{abstract}

\maketitle

%% start the paper here:
\includepdf[pages=-,offset=0 0.7in,pagecommand={}]{sample-word.pdf}
%\newpage
%Lorem ipsum
%\newpage
%Lorem ipsum
\end{document}
